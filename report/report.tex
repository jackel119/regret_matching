\documentclass [11pt]{article}
\usepackage{fullpage}
\usepackage[utf8]{inputenc}
\usepackage{indentfirst}
\usepackage{listings}
\usepackage{xcolor}
\usepackage{titlesec}
\usepackage[shortlabels]{enumitem}
\usepackage{graphicx}
\usepackage[backend=bibtex,style=numeric]{biblatex}
\usepackage[left=1.75in, right=1.75in, top=1in, bottom=1in]{geometry}
\usepackage{amsmath,amsthm,amssymb}
\usepackage{bm}
\usepackage{MnSymbol}
\usepackage{mathtools}
\usepackage{tikz}
\usetikzlibrary{bayesnet}
\usepackage{float}
\usepackage{caption}
\usepackage{subcaption}
\usepackage{appendix}

\graphicspath{ {../images/} }

\newcommand{\ra}{\rightarrow}
\DeclarePairedDelimiter{\abs}{\lvert}{\rvert}

\setlength{\jot}{10pt}
\title{Dynamics of Games Project\\
  Regret Minimalization of Colonel Blotto Game
}
\author{Pobpawat Pordi \\
  CID: 01192761 \\
}

\begin{document}

\maketitle

\pagebreak

\tableofcontents

\pagebreak

\section{Introduction}

This paper will focus on the task described by Project 3 of the Dynamics of Games offered by the Department of Mathematics at Imperial College.

First, we will take a brief look at the game of Colonel Blotto described in TODO (specifically, the case of 5 armies and 3 battlefields) and form some predictions about the any equilibria TODO and/or possible dynamics the game may have. In section \ref{algorithm} we will then describe our customized implementation of the regret-matching algorithm from the Neller-Lancot paper, as well as other tools and utilities that are of use in analysis of iterated one-shot games. Section \ref{algo-theory} will describe the results of the Colonel Blotto game achieved by the algorithm, and

\section{Colonel Blotto Game} \label{blotto game}

The Colonel Blotto game is a two-player, one-shot, zero-sum game revolving around the idea of two opposing players each having $a$ discrete armies and having to split them up to fight on $b$ different battlegrounds. Each battlegrounds is won by a player if and only if they have sent a larger force to that battleground, a draw if both players have sent the same number of armies, and a loss otherwise. The player who has won more battlegrounds wins the game, and the other loses. A draw is possible if neither player has won more battlegrounds than the other. The utility for a win, draw, and loss is 1, 0, and -1 respectively.

For example, in the case of each player having 4 armies and 3 battlefields, player I may choose to split their armies as $(2, 2, 0)$ - meaning 2 armies to the firt battlefield, 2 armies to the 2nd battlefield, and no armies to the third battlefield. Player II may choose to utilise their army via $(0, 1, 3)$. The result of the game is that player I's armies are victorious in the first two battlefields, and player II only wins the third battleground, therefore player I wins and is awarded a utility of 1, and player II gains a utility of -1.

Analysis of this game is a tricky, for mainly for two reasons:

\begin{enumerate}
  \item The number of possible actions for $a$ armies and $b$ battlefields is $\frac{(a + b - 1)!}{a! (b -1)!}$, meaning the size of the payoff matrix grows very quickly with $a$ and $b$ , and even for small cases such as $a=4, b=2$  have a $10x10$ payoff matrix.
  \item Small cases which are feasable to analyse by hand are uninteresting. For example, with $a = 1, b = 2$, the payoff matrix is actually a 2x2 matrix of zeroes, as neither player can win. In fact for all cases of $b=2$ neither player can ever win.
\end{enumerate}

\subsection{Symmetric Colonel Blotto}

Note that the actions in any Colonel Blotto game are \textit{symmetric}. That is, all actions have corresponding actions such that the payoffs are equivalent. For example, $(a_1, a_2, a_3)$ has the same payoff against $(b_1, b_2, b_3)$ as $(a_2, a_1, a_3)$ has to $(b_2, b_1, b_3)$, and so forth.

We may then first attempt to analyze a simpler, symmetric version of the Colonel Blotto game, called the \textit{symmetric} Colonel Blotto. Here, we group actions of the original Colonel Blotto game together by their permutative equivalences, e.g. an action $a_1 a_2 a_3 $ represents all of its possible permutations in the original Colonel Blotto game. We assume each permutation of an action is equally likely due to symmetry, therefore we can calculate the payoff matrix of the Symmetric Colonel Blotto game with 3 armies and 5 battlefields.

\begin{figure}[H]
  \renewcommand{\arraystretch}{1.75}
  \centering
  \begin{tabular}{|c|c|c|c|c|c|}
  \hline
  & \texttt{005} & \texttt{014} & \texttt{023} & \texttt{113} & \texttt{122} \\[2ex]
  \hline
    \texttt{005} & 0 & $- \frac{1}{3}$ & $- \frac{1}{3}$ & $-1$ & - 1 \\
    \hline
    \texttt{014} & $\frac{1}{3}$ & 0 & 0 & $-\frac{1}{3}$ & $- \frac{2}{3}$ \\
    \hline
    \texttt{023} & $\frac{1}{3}$ & 0 & 0 & 0 & $ \frac{1}{3}$ \\
    \hline
    \texttt{113} & 1 & $\frac{1}{3}$ & 0 & 0 & $ -\frac{1}{3}$ \\
    \hline
    \texttt{122} & 1 & $\frac{2}{3}$ & $-\frac{1}{3}$ & $-\frac{1}{3}$ & 0 \\
    \hline
  \end{tabular}
  \caption{Payoff for Symmetric Colonel Blotto}%
  \label{sym-blotto-full}
\end{figure}

Each element of the matrix is the expected payoff of any (uniform) random permutation of the row action against any (uniform) random permutation of the column action.

\subsubsection{Nash Equilibria}

Recall the definition of Nash Equilibria in a 2-player game:

\begin{align}
  \begin{split}
    & x \cdot A \hat{y} \leq \hat{x} \cdot A \hat{y} \\
    & y \cdot B \hat{x} \leq \hat{y} \cdot B \hat{x}
  \end{split}
\end{align}

or equivlalently in terms of \textit{best responses}

\begin{align}
  \begin{split}
    & \hat{x} \in \mathcal{BR}_A( \hat{y} ) \\
    & \hat{y} \in \mathcal{BR}_B( \hat{x} )
  \end{split}
\end{align}

where $ \mathcal{BR}_A(y)  = \{ x^* \in \Delta \; | \; x^* \cdot A y \geq x \cdot A y \; \forall x \in \Delta  \} $.\\

Note that in the payoff matrix, the first two rows always perform worse than the bottom 3 rows, regardless of the column - i.e. actions \texttt{005} and \texttt{014} are dominated by the other actions. Hence no Nash Equilibria would involve those two actions, and we can consider the sub-matrix of the last 3 actions (with the payoffs normalized).

\begin{align}
  \begin{bmatrix}
    0 & 0 & 1 \\
    0 & 0 & -1 \\
    -1 & 1 & 0 \\
  \end{bmatrix}
\end{align}

For the above matrix, there are 4 Nash Equilibria:

\begin{align}
  \hat{x} = (1, 0, 0)^T, \; \hat{y} = ( 1, 0, 0 )^T \\
  \hat{x} = (1, 0, 0)^T, \; \hat{y} = (\tfrac{1}{2}, \tfrac{1}{2}, 0)^T \\
  \hat{x} = (\tfrac{1}{2}, \tfrac{1}{2}, 0)^T, \; \hat{y} = (1, 0, 0)^T \\
  \hat{x} = (\tfrac{1}{2}, \tfrac{1}{2}, 0)^T, \; \hat{y} = (\tfrac{1}{2}, \tfrac{1}{2}, 0)^T
\end{align}

To show that the above are indeed Nash Equilibria, we show that each $\hat{x}$ we and $\hat{y}$ are each others' best reponses.



\section{Implementation of Regret Minimalization Algorithm and Other Utilities} \label{algorithm}

Blah blah

\section{Results of Algorithm} \label{results}

Blah blah

% \begin{figure}[H]
%   \centering
%   \includegraphics[width=0.7\linewidth]{PCA.jpg}
%   \caption{Accuracy of PCA}
% \end{figure}

% \begin{figure}[H]
%   \centering
%   \includegraphics[width=0.7\linewidth]{wPCA.jpg}
%   \caption{Accuracy of Whitened PCA}
% \end{figure}

% \begin{figure}[H]
%   \centering
%   \includegraphics[width=0.7\linewidth]{LDA.jpg}
%   \caption{Accuracy of LDA}
% \end{figure}


\end{document}
